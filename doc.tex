\documentclass{report}

\newcommand{\name}{\textit{Arcana Azurea Neferpitou}}
\newcommand{\program}{\textit{NEKO}}

\usepackage[french]{babel}
\usepackage[T1]{fontenc}
\usepackage[utf8]{inputenc}
\usepackage{fontspec}
\usepackage[a4paper]{geometry}
\usepackage{scrextend}
\usepackage{listings}
\usepackage[hidelinks,unicode=true]{hyperref}
\usepackage[pgf]{dot2texi}
\usepackage{pgf}
\usepackage{tikz}
\usepackage{fancyhdr}
\usepackage{sectsty}
\usepackage{titlesec}

\titleformat{\chapter}[display]
  {\Huge}
  {\filleft\texttt{\chaptertitlename} \Huge\thechapter}
  {0ex}
  {\filleft}
  [\titlerule]

\usetikzlibrary{automata}

\pagestyle{fancy}
\fancyhf{}

\lhead{\leftmark}
\rhead{\rightmark}
\rfoot{\thepage}

\chapterfont{\raggedleft}

\setmainfont[
    Path = fonts/sazanami-neko/,
    Extension = .ttf,
    Ligatures = TeX,
    Scale = MatchLowercase,
]{sazanami-mincho}

\setsansfont[
    Path = fonts/sazanami-neko/,
    Extension = .ttf,
    Ligatures = TeX,
    Scale = MatchLowercase,
]{sazanami-gothic}

\title{"neko"}

\author{
   adjivas
   \and
   brezaire
   \and
   flime
   \and
   jpepin
}

\date{}

\begin{document}

\tableofcontents

\chapter{Premiere partie}

\section{Préambule}

\begin{minipage}{1in}
  \fontsize{50pt}{12pt}\selectfont
  \symbol{"E000}\symbol{"E001}\symbol{"E002}\symbol{"E003}\symbol{"E004}\symbol{"E005}\symbol{"E006}\symbol{"E007}\symbol{"E008}\symbol{"E009}\\*
  \symbol{"E00A}\symbol{"E00B}\symbol{"E00C}\symbol{"E00D}\symbol{"E00E}\symbol{"E00F}\symbol{"E010}\symbol{"E011}\symbol{"E012}\symbol{"E013}\\*
  \symbol{"E014}\symbol{"E015}\symbol{"E016}\symbol{"E017}\symbol{"E018}\symbol{"E019}\symbol{"E01A}\symbol{"E01B}\symbol{"E01C}\symbol{"E01D}\\*
  \symbol{"E01E}\symbol{"E01F}\symbol{"E020}\symbol{"E021}\symbol{"E022}\symbol{"E023}\symbol{"E024}\symbol{"E025}\symbol{"E026}\symbol{"E027}\\*
  \symbol{"E028}\symbol{"E029}\symbol{"E02A}\symbol{"E02B}\symbol{"E02C}\symbol{"E02D}\symbol{"E02E}\symbol{"E02F}\symbol{"E030}\symbol{"E031}\\*
\end{minipage}

Une nékoe est un persona d'animé japonais avec des traits de chat
\textendash{ mimikko
	\footnote{ Kemonomimi ou mimikko est un personnage humain d'animé avec des caractéristiques animales tel que la personnalité ou encore le physique
		\textendash{ 獣耳 }\textendash.
			}} \textendash.

Le GlyphArt est l'écriture d'une image via des caractères compris dans l'Unicode privé, ce projet démontre ce procédé via
\href{https://limaconoob.github.io/Image2font}{Image2font}.

$\name$ est une programmeuse nékoe de fiction inventé pour assister son utilisateur.

\section{Introduction}
\thispagestyle{empty}
$\program$ est un prompt nommé $\name$ et qui apportera les Arts, la Culture et son assistance a qui saura utiliser un shell.
Humanisée d’émotions et fondée sur l'experience de la chambre chinoise, celle-ci sera donc instruite via des biblioteques.

La liste des options est :
\begin{labeling}{Longer label\quad}
	\item[\textbf{
		\textendash p,
		\textendash\textendash from-part <file.neko.part, ...>}] innitialise une liste de 
			\textendash{texels}
				\footnote{ Un texel ou élément de texture est l'unité fondamental de mesure d'une texture. }.
	\item[\textbf{
		\textendash s,
		\textendash\textendash from-sprite <file.neko.sprite, ...>}] innitialise une liste de
			\textendash{sprite}
				\footnote{ Un sprite est une texture délimitait par une surface. }.
\end{labeling}

La liste des commandes est :

\begin{labeling}{Longer label\quad}
	\item[\textbf{
		\textendash m,
		\textendash\textendash mount <[<name, link, object>, ...]>
	}] monte dynamiquement une liste de biblioteques depuis un paquet, un dépôt Git ou un fichier.
	\item[\textbf{
		\textendash c,
		\textendash\textendash config,
		\textendash\textendash configuration <name>
	}] initialise le programme avec une nouvelle liste de biblioteques.
	\item[\textbf{
		\textendash s,
		\textendash\textendash sprite <position> [<attribut>, ...]
	}] change l'expression de $\name$ .
\end{labeling}

\subsection{Module Graphique}

Ce module est l'interface d'une liste de primitives telles que des texels et des sprites qui ponrront ce combiner en de nouveaux sprites.

\begin{figure}[!h]
\centering
  \begin{dot2tex}[dot,scale=0.35]
digraph UML {
  d2tstyleonly = true;
  node [shape = "box", width = "4.5", texmode="verbatim", style = "top color=cyan!10,bottom color=cyan!35,draw=cyan!50,rounded corners"];
  graph [shape = "box", texmode="math", style = "top color=blue!10,bottom color=blue!35,draw=blue!50"];

  nodePartError [label="Enum PartError<Display + Debug + Error>\n
    UnknownPart
    ForbiddenGlyph(char)
  "];

  nodePart [label="Enum Part\n
    EyeLeft(char)
    EyeRight(char)
    EarLeft(char)
    EarRight(char)
    Nose(char)
    Mouth(char)
    Neck(char)
	ShoulderLeft(char)
	ShoulderRight(char)\
    \n
    + new(limb: &str, glyph: char) -> Result<Self>
  "];

  nodeManagerError [label="Enum ManagerError<Display + Debug + Error>\n
      Duplicate
      BadPart(PartError)
  "];

  nodeManager [label="Struct Manager<Default>\n
    part: HashMap<(Posture, Emotion), Part>
	sprite: HashMap<String, Sprite>\
	\n
    + insert_part(&mut self, key: (Posture, Emotion), value: Part) -> Result<()>
	+ insert_sprite(&mut self, key: String, value: Sprite) -> Result<()>
  ", width = 8.5];

  subgraph clusterPostureEmotion {
    nodePosture [label="Enum Posture\n
      LotusHandsOnFloor
      LyingOnSomething
    "];
    nodeEmotion [label="Enum Emotion\n
      Happy
      Malicious
      None
    "];
  }

  nodeSpriteError [label="Enum SpriteError<Display + Debug + Error>\n
    EmptyBoard
  "];

  nodeSprite [label="Struct Sprite\n
    const MAX_X: usize = 7;
    const MAX_Y: usize = 10;
	interval: usize
    sheet: Vec<(Posture, [[(Emotion, Part); MAX_X]; MAX_Y])>\
    \n
    + new(sheet: Vec<(Posture, [[(Emotion, Part); MAX_X]; MAX_Y])>, interval: usize) -> Result<Self>
  ", width = 8.5];

  nodePartError -> nodePart;
  nodePart -> nodeSprite;
  nodePart -> nodeManager;
  nodeManagerError -> nodeManager;
  nodePosture -> nodeSprite;
  nodePosture -> nodeManager;
  nodeEmotion -> nodeSprite;
  nodeEmotion -> nodeManager;
  nodeSpriteError -> nodeSprite;
  nodeSprite -> nodeManager;
}
  \end{dot2tex}
  \caption{Diagramme du module graphique.}
  \label{UML}
\end{figure}

\end{document}
