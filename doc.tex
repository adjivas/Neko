\documentclass[french]{article}

\newcommand{\name}{\textit{Arcana}}
\newcommand{\program}{\textit{NEKO}}

\usepackage[utf8]{inputenc}
\usepackage[T1]{fontenc}
\usepackage{lmodern}
\usepackage[a4paper]{geometry}
\usepackage{babel}
\usepackage{scrextend}
\begin{document}

\tableofcontents

\section{Préambule}

Une nékoe est-un caractère d'animé japonais avec des traits de chat -mimikko-.

\addcontentsline{toc}{section}{Unnumbered Section Préambule}

\section{Introduction}

$\program$ est-un prompt nommée $\name$ et qui apportera les Arts, la Culture et son assistance a qui saura utiliser un shell.
Humanisée d’émotion et fondé sur l’experience de la chambre chinoise, celle-si sera donc instruite via des biblioteques.

La liste des commandes est :

\begin{labeling}{Longer label\quad}
\item[Mount] monte dynamique une biblioteques.
\item[Emotion] change l'expression de $\name$.
\end{labeling}

\addcontentsline{toc}{section}{Unnumbered Section Introduction}

\end{document}
