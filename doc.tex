\documentclass{book}

\newcommand{\name}{\textit{Arcana Azurea}}
\newcommand{\program}{\textit{NEKO}}

\usepackage[utf8]{inputenc}
\usepackage[french]{babel}
\usepackage[T1]{fontenc}
\usepackage{lmodern}
\usepackage{fontspec}
\usepackage[a4paper]{geometry}
\usepackage{scrextend}
\usepackage{listings}
\usepackage[unicode=true]{hyperref}
\usepackage[pgf]{dot2texi}
\usepackage{pgf}
\usepackage{tikz}

\usetikzlibrary{automata}

\setmainfont[
    Path = fonts/sazanami-neko/,
    Extension = .ttf,
    Ligatures = TeX
]{sazanami-mincho}

\setsansfont[
    Path = fonts/sazanami-neko/,
    Extension = .ttf,
    Ligatures = TeX
]{sazanami-gothic}

\begin{document}

\tableofcontents

\chapter{Premiere partie}

\section{Préambule}

\begin{lstlisting}
\
\
\
\
\
\end{lstlisting}

Une nékoe est un persona d'animé japonais avec des traits de chat
\textendash{ mimikko
	\footnote{ Kemonomimi ou mimikko est un personnage humain d'animé avec des caractéristiques animales tel que la personnalité ou encore le physique
		\textendash{ 獣耳 } \textendash.
			}} \textendash.

Le GlyphArt est l'écriture d'une image via des caractères compris dans l'Unicode privé, ce projet démontre ce procédé via
\href{https://limaconoob.github.io/Image2font}{Image2font}.

$\name$ est une programmeuse nékoe de fiction inventé pour assister son utilisateur.

\section{Introduction}
\thispagestyle{empty}
$\program$ est un prompt nommé $\name$ et qui apportera les Arts, la Culture et son assistance a qui saura utiliser un shell.
Humanisée d’émotions et fondée sur l'experience de la chambre chinoise, celle-ci sera donc instruite via des biblioteques.

La liste des commandes est :

\begin{labeling}{Longer label\quad}
	\item[\textbf{-m, --mount <[<name, link, object>, ...]>}] monte dynamiquement une liste de biblioteques depuis un nom, un lien ou un objet.
	\item[\textbf{-c, --config, --configuration <name>}] initialise le programme avec une nouvelle liste de biblioteques.
	\item[\textbf{-s, --sprite <position> [<attribut>, ...]}] change l'expression de $\name$ .
\end{labeling}

\section{NEKO}

\begin{dot2tex}[dot]
digraph UML {
	d2tstyleonly = true;
	node [shape = "box", width = "3", texmode="math", style = "top color=cyan!10,bottom color=cyan!35,draw=cyan!50,rounded corners"];

	node0 [label="2:A<1>"];

	nodePosition [label="Position\n
		hair:Option{<}Attribut{>}
		eyes:Option{<}Attribut{>}
		nose:Option{<}Attribut{>}
		mouth:Option{<}Attribut{>}
		neck:Option{<}Attribut{>}
		tail:Option{<}Attribut{>}
		foot:Option{<}Attribut{>}
	"];

	nodeSprite [label="Trait Sprite\n
		
	"];

	node0 -> nodePosition;

}
\end{dot2tex}

\end{document}
