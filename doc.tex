\documentclass[french]{article}

\newcommand{\name}{\textit{Arcana}}
\newcommand{\program}{\textit{NEKO}}

\usepackage{fontspec}
\usepackage[utf8]{inputenc}
\usepackage[T1]{fontenc}
\usepackage{lmodern}
\usepackage[a4paper]{geometry}
\usepackage{babel}
\usepackage{scrextend}
\usepackage[]{dot2texi}
\usepackage{tikz}

\usetikzlibrary{shapes,arrows}
\begin{document}

\tableofcontents

\section{Préambule}

Une nékoe est un caractère d'animé japonais avec des traits de chat
\textendash{ mimikko
	\footnote{ Kemonomimi ou mimikko est un personnage humain d'animé avec des caractéristiques animales tel que la personnalité ou encore le physique
		\textendash{ 獣耳 } \textendash.
			}} \textendash.

Le GlyphArt est l'écriture d'une image via des caractères compris dans l'Unicode privé

\addcontentsline{toc}{section}{Unnumbered Section Préambule}

\section{Introduction}

$\program$ est un prompt nommée $\name$ et qui apportera les Arts, la Culture et son assistance a qui saura utiliser un shell.
Humanisée d’émotion et fondé sur l’experience de la chambre chinoise, celle-si sera donc instruite via des biblioteques.

La liste des commandes est :

\begin{labeling}{Longer label\quad}
	\item[Mount] monte dynamiquement une biblioteques.
	\item[Emotion] change l'expression de $\name$.
\end{labeling}

\begin{dot2tex}[neato,mathmode]
digraph G {
	node [shape="circle"];
	a_1 -> a_2 -> a_3 -> a_4 -> a_1;
}
\end{dot2tex}

\addcontentsline{toc}{section}{Unnumbered Section Introduction}

\end{document}
